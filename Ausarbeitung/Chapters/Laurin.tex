\section{Ursprung}

Wie im Artikel zu \ac{DevOps} auf den Seiten 10 und folgende des Magazins Objektspektrum 06/20 \cite{spektrum1} kurz angesprochen, beruht die Idee von \ac{DevOps} ursprünglich nicht auf Prinzipien aus der IT, sondern Prozessideen aus der industriellen Anfertigung, vor allem aus den Produktionshallen von Toyota. Von dort werden auch die Ansätze des Lean Managements und Kanban genutzt, die beide die Prozesskette als sehr wichtig betrachten und als Grundidee haben, dass sich die einzelnen Teilprozesse untereinander verständigen und abstimmen, um insgesamt den Prozess effizienter und stabiler zu betreiben. Insgesamt ging es bei Toyota damals im Bezug auf die effizientere Gestaltung der Prozesskette darum, alles darauf auszulegen, dass die Zeitspanne zwischen der Erteilung eines Auftrags bis zu dem Zeitpunkt, an dem das Geld fließt, so stark verkürzt wird wie möglich. \cite{halstenberg:2020} Das ist unter anderem durch weglassen unnötigen Aufwands (im japanischen \glqq Muda\grqq\ genannt) und die effizientere Gestaltung einzelner Teilprozesse durch die Nutzung der Ideen von Kanban und Lean zu erreichen. Kanban beschreibt grundsätzlich das Konzept, dass jeder Teilprozess eigenständig seine Produktion verwaltet, d. h. jede Abteilung fordert selbst an, wie viele Ressourcen benötigt werden und auf der anderen Seite produziert jede Abteilung nur genau so viel, wie die folgende Abteilung anfordert. Dadurch wird eine zentralisierte Ressourcenverteilung in den Hintergrund gestellt, und jede Abteilung ist selbst verantwortlich, das sie grundsätzlich am besten weiß, was gerade wie viel benötigt wird \cite{ohno:1988}. Die Ideen des Lean Managements werden in den folgenden Abschnitten angesprochen und vor allem in \autoref{sec:calms} genauer beschrieben.

\section{Leitsatz}

\section{CALMS}\label{sec:calms}

\section{Guiding Tools}