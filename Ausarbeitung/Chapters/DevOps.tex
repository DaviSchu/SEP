\chapter{DevOps}

\section{Ursprung}

Wie im Artikel zu \ac{DevOps} auf den Seiten 10 und folgende des Magazins Objektspektrum 06/20 \cite{spektrum1} kurz angesprochen, beruht die Idee von \ac{DevOps} ursprünglich nicht auf Prinzipien aus der IT, sondern Prozessideen aus der industriellen Anfertigung, vor allem aus den Produktionshallen von Toyota. Von dort werden auch die Ansätze des Lean Managements und Kanban genutzt, die beide die Prozesskette als sehr wichtig betrachten und als Grundidee haben, dass sich die einzelnen Teilprozesse untereinander verständigen und abstimmen, um insgesamt den Prozess effizienter und stabiler zu betreiben. Insgesamt ging es bei Toyota damals im Bezug auf die effizientere Gestaltung der Prozesskette darum, alles darauf auszulegen, dass die Zeitspanne zwischen der Erteilung eines Auftrags bis zu dem Zeitpunkt, an dem das Geld fließt, so stark verkürzt wird wie möglich. \cite{halstenberg:2020} Das ist unter anderem durch weglassen unnötigen Aufwands (im japanischen \glqq Muda\grqq\ genannt) und die effizientere Gestaltung einzelner Teilprozesse durch die Nutzung der Ideen von Kanban und Lean zu erreichen. Kanban beschreibt grundsätzlich das Konzept, dass jeder Teilprozess eigenständig seine Produktion verwaltet, d. h. jede Abteilung fordert selbst an, wie viele Ressourcen benötigt werden und auf der anderen Seite produziert jede Abteilung nur genau so viel, wie die folgende Abteilung anfordert. Dadurch wird eine zentralisierte Ressourcenverteilung in den Hintergrund gestellt, und jede Abteilung ist selbst verantwortlich, das sie grundsätzlich am besten weiß, was gerade wie viel benötigt wird \cite{ohno:1988}. Die Ideen des Lean Managements werden in den folgenden Abschnitten angesprochen und vor allem in \autoref{sec:calms} genauer beschrieben.

\section{Leitsatz}

\section{CALMS}\label{sec:calms}

\section{Guiding Tools}

\section{Fehlerkulturen \cite{spektrum2}}\label{chap:fehler}

Die Fehlerkulturen beschreiben die grundlegenden Arten auf die mit Fehlern umgegangen werden kann. Es geht darum als was Fehler betrachtet werden, also als etwas Gutes oder Schlechtes, ob sie Strafe mit sich bringen, ein Risiko darstellen und so weiter.

Es gibt prinzipiell innerhalb \ac{DevOps} 2 Fehlerkulturen die im Konflikt zueinander stehen. Diese sehen wie folgt aus:

\begin{description}
\item[Nullfehlertoleranz]\hfill\\ Wie der Name bereits vermuten lässt, sollen hierbei Fehler absolut ausgeschlossen sein. Sie werden immer als etwas Negatives betrachtet, da sie ein Unternehmen Reputation oder Geld kosten können. In schweren Fällen und je nach Anwendungsgebiet, kann ein Fehler sogar ein Leben kosten. Daher ist es unerlässlich, eine solche Nulltoleranz in manchen Fällen walten zu lassen. Folglich sind die Fehler auch oft mit Scham behaftet, und bedeuten für Denjenigen der ihn zu verantworten hat oft eine gewisse Strafe. Dies übt einen großen Druck aus und ist in der Realität nie zu hundert Prozent zu gewährleisten.

Diese strikte Umgangsart mit Fehlern regt natürlich auch dazu an, möglichst bei Altbekanntem zu bleiben, und keine Experiment zu wagen. Dabei ist natürlich kein großer Raum für Fortschritt.

Um diese Freiheit von Fehlern zu gewährleisten, fallen oft auch weitere Kosten an, beispielsweise für weitere spezialisierte Tools, oder mehr Personentage um ausgiebiger testen zu können. Andererseits sollen hierbei auch einfach Missverständnisse und Fehler durch den letztendlichen Anwender vermieden werden, der vielleicht das Interface nicht eindeutig oder intuitiv genug zu bedienen findet. Stichwort Human Computer Interaction.


\item[Experimente erlauben]\hfill\\ Auch diese Begrifflichkeit ist selbsterklärend. Fehler sind hierbei nicht nur weniger schlimm als bei der Nullfehlertoleranz, sie werden sogar offen begrüßt und sind erwünscht. Ein Fehler ist hierbei nicht typischerweise schambehaftet.

Es geht um ein Trial-and-Error-Mindset, welches zum Experimentieren anregen soll. Denn nur das Experimentieren mit neuen Technologien oder Ideen, bringt auch tatsächlich Fortschritt und kann somit auf langfristige Sicht einen großen Vorteil bedeuten. Denn es kann stets auf neue Entwicklungen und Technologien reagiert werden. \glqq Handeln kann Fehler verursachen; nicht zu handeln kann der größere Fehler sein.\grqq
\end{description}

Die Kunst besteht nun darin, die richtige Fehlerkultur an der richtigen Stelle zu wählen. Sie stehen sich ganz klar gegensätzlich im Weg und können nicht miteinander vereint werden. Daher muss immer abgeschätzt werden, wie schwerwiegend Fehler an einer bestimmten Stelle sein können. Wenn es beispielsweise darum geht, dass ein bemanntes Flugzeug abstürzen könnte wenn ein System ausfällt, dann wird die Nullfehlertoleranz gewählt. Hier werden eindeutig keine Fehler gewünscht und solche können dramatische Folgen haben. In einer kleineren Verwaltungssoftware zum Beispiel hingegen, sind Fehler weniger gravierend und leichter zu beheben. Hier kann also zu Gunsten einer möglichen Verbesserung von Prozessen oder des Systems in der Zukunft experimentiert werden. So eindeutig sind die Fälle jedoch nicht immer. Daher muss der Grad zwischen Verbesserungspotential und möglichen Schäden durch Fehler abgewägt werden.

\section{Kritikpunkte}

Die folgenden Kritikpunkte und ihre Bewertungen wurden so von \citeauthor{halstenberg:2020} \cite{halstenberg:2020} präsentiert. Die Autoren betonen, dass \ac{DevOps} kein Selbstläufer ist, und vor allem aber auch dass die Idee von \ac{DevOps} gestärkt aus den Kritiken hervor geht.

\subsection{Nur ein Hype}

Einer der größten, und vielleicht der häufigste Kritikpunkt war die Annahme, dass \ac{DevOps}, genauso wie unzählige andere Hypes in der IT, schon bald an Relevanz verlieren wird und sich herausstellt, dass doch kein größerer Nutzen dahintersteckt. Allerdings hat sich gezeigt dass sich die Nutzung und die Präsenz des Themas \ac{DevOps} nicht verringert hat, sondern im Gegenteil immer fester verankert in der heutigen Unternehmenswelt steht. Auch Lean, das dem jetzigen \ac{DevOps} zugrunde liegt, hat sich fest in der Fertigung verankert und ist ein zentraler Bestandteil des Prozesses geworden. Selbiges ist von \ac{DevOps} auch zu erwarten. Die \ac{ITIL} ist der Defacto-Standard wenn es um Service Management geht. Sie stellt einen Best-Practice-Leitfaden dar und hier wurden bereits \ac{DevOps} Anregungen und Konzepte aufgenommen.

\subsection{DevOps funktioniert nicht}

2018 wurde hierzu von Heise ein Artikel veröffentlicht mit dem Titel \glqq \ac{DevOps} funktioniert nicht\grqq \cite{weiss:2018}. Es geht darum, dass \ac{DevOps} seine Versprechen nicht halten kann und in der Praxis einfach nicht zu den gewünschten Erfolgen führt.

Grund dafür ist jedoch nicht, dass \ac{DevOps} an sich nicht funktioniert, sondern dass es, laut dem Artikel, am Management scheitert. Dies ist ein großes Missverständnis das im Kontext zu \ac{DevOps} herrscht. Oft wird davon ausgegangen, dass \ac{DevOps} kaufbar ist. Man stellt jemanden dafür ein, installiert etwas, und dann klappt es. Dies entspricht jedoch nicht der Realität, aus dem einfachen Grund dass es sich bei \ac{DevOps} nicht um ein Werkzeug handelt, sondern viel mehr um die Denkweise und die persönliche Einstellung eines jeden Involvierten.

Wenn das Denken nicht nachhaltig auf \ac{DevOps} angepasst wird, und sich vielleicht nach ein paar Wochen oder Monaten die zum Ziel gesetzten Leitsätze verlaufen, dann ist das gesamte Vorhaben zum Scheitern verurteilt. Dies ist laut einer Studie der \ac{IDC} \cite{idc:2020} auch der am häufigsten angegebene Grund des Scheiterns, beziehungsweise das größte Hindernis bei der Einführung von \ac{DevOps}. Die Beharrlichkeit auf bestehenden Konzepten und der Widerstand gegen Veränderung.

\subsection{Silos sind jetzt Andere}

Ein weiterer der prominenteren Vorwürfe ist der, dass es die Silos jetzt immer noch gibt, allerdings diese jetzt anders verteilt sind. Es gibt also nicht mehr die vertikalen Silos über beispielsweise Abteilungen, sondern horizontale, die dann beispielsweise Teams sind die aus Mitarbeitern der unterschiedlichen Abteilungen bestehen. Auch hier sei die Gefahr enorm, dass die Teams gegeneinander arbeiten und sich weiterhin am liebsten gegenseitig die Schuld zuschieben.

Wie bereits beim letzten Kritikpunkt ist auch hier das Problem wieder nicht das Prinzip hinter \ac{DevOps}, sondern viel mehr das Denken dahinter. Wenn ein Unternehmen es nicht schafft, \ac{DevOps} in den Köpfen seiner Mitarbeiter zu etablieren, dann wird man damit auch keinen Erfolg haben, da die Einführung neuer Methoden, oder die Aufteilung in neue Teams, als pure Umstrukturierung missverstanden werden kann. \glqq Es gilt die alte Weisheit: gib einem Team zwei verschiedene Leibchen und sie werden gegeneinander spielen. Mit \ac{DevOps} versuchen wir nicht, die zwei Leibchen anders zu verteilen sondern alle mit gleichen Leibchen füreinander spielen zu lassen\grqq \cite{halstenberg:2020}

\subsection{DevOps ist Cargo Cult}

Mit Cargo-Cult beschreibt man ganz knapp das Nachahmen von Handlungen, die mit fremdem Erfolg in Verbindung gebracht werden können. Man sieht beispielsweise dass ein anderes Unternehmen großen Erfolg hat, und führt dies auf einzelne Teilprozesse zurück die sich von den Eigenen unterscheiden. So gelangt man an einen Punkt, an dem Praktiken imitiert werden, die aber in der eigenen Prozesskette sinnlos erscheinen.

Die Wortherkunft führt auf den zweiten Weltkrieg zurück, bei dem beobachtet wurde wie die indigenen Völker unter Anderem Fluglotsen nachahmten, nachdem sie diese von der amerikanischen Armee gesehen hatten, immer wenn wertvolle \textit{Cargo} mit einem Flugzeug kam. Die Gefahr in diese Falle zu laufen, in der man ohne Erfolgsaussichten Handlungen und Prinzipien nachahmt, geht davon aus dass sich auf den Erfolgsfaktor verlassen wird, ohne die Tauglichkeit, oder die Verbindung zur eigenen Sache überprüft zu haben. \ac{DevOps} muss wie alles Andere auch, auf ein Unternehmen zugeschnitten werden, und kann nicht blind kopiert werden.

Wichtig bei der Einführung von \ac{DevOps} ist es, dass auch etwas Unangenehmes wie über Misserfolge sprechen auf der Tagesordnung steht. Es muss stetig geprüft werden, wie sich die Etablierung von \ac{DevOps} entwickelt, und wo vielleicht nachgebessert werden muss.

\subsection{DevOps und Regulatorik passen nicht zusammen}

Des weiteren wird \ac{DevOps} nachgesagt, dass es einfach nicht funktioniert unter gewissen Bedingungen. Im Buch wird das Beispiel des Bankenbereiches aufgebracht, der wie viele andere Bereiche in der IT strengen Regelungen unterliegt. Dabei sind \glqq Mindesterwartungen zu Anforderungsdokumentation,
Dokumentation von Entwicklungsprozessen, zum Berechtigungsmanagement, zum Testing, etc. festgelegt\grqq \cite{halstenberg:2020}.

Die Autoren sehen hierin allerdings keinen Widerspruch und finden nicht dass \ac{DevOps} unter solchen Rahmenbedingungen leidet. Im Gegenteil, denn \ac{DevOps} sei kein Freibrief um über das Fehlen von Prozessen, Tests oder Dokumentation hinwegzusehen.

\subsection{Bei DevOps erfolgt der Betrieb aus der Entwicklung heraus, mit der Cloud ist Ops überfüssig}

Abschließend ein weiteres Missverständnis. Dieses Denken will mit \ac{DevOps} überwunden werden, denn es birgt enorme \glqq operative Risiken\grqq \cite{halstenberg:2020}.
